\chapter{Conclusion and Discussion}
\label{chap:ConAndDis}

%\section{Conclusion}
The long-term high temporal resolution CF-IRMS measurements showed high-concentration methane peaks. Those peaks were observed for the total measurement campaign over eight months. The FTIR network measurements performed for a shorter time period also were able to observe those peaks at favourable weather conditions. Due to additional measurement limitations of the FTIR sensor network, a continuous measurement was not possible, excluding measurements during the night and in cloudy conditions. This resulted in only very few instances where methane peaks could be observed with both measurement approaches simultaneously, prohibiting a detailed direct comparison between the two approaches.\\
A Keeling plot analysis of the CF-IRMS time series showed an isotopic signature of the methane in the air that was strongly influenced by natural emitters. While measurements performed during northern wind directions were shifted more to emission types that are observed with Anthropogenic emitters. To the north of the measurement site, mostly residential areas are located. Previous studies in this region showed a high methane enrichment with numerous plumes, most likely originating from natural gas infrastructure leakages \cite{Maazallahi.2020}. A negative correlation between the measured methane concentration with the CF-IRMS measurements and the air temperature was identified for that residential region. This further indicates that a substantial amount of methane originating from this region originated from anthropomorphic sources such as fossil fuel burning. At colder temperatures, the consumption of those fuels increases for heating and private transport, resulting in elevated methane releases to the atmosphere from incomplete combustion of natural gas and leakages from consumer products \cite{Lebel.2022}.\\
CF-IRMS measurements performed at wind directions from the South were observed to have an isotope composition mainly from Natural emitters, such as wetlands, waste treatment and agriculture. To the south of the measurement location, the Elbe and the industrial port region of Hamburg were located. In this region, it was found with the aid of the Bayesian inversion modelling that the Elbe is a major methane emitter, emitting 1900 $\pm$ 1000  kg h$^{-1}$ of methane. \\
An additional possible contributor in the Hamburg Port region is the presence of civil infrastructure, which includes wastewater treatment, garbage processing, etc. Those locations have been investigated closely with the mobile drive-by measurements using a boat and a car. By this method, no significant elevated methane concentrations have been observed near those locations \cite{Forstmaier.2023}. Further questioning of the facility's operators revealed that the facilities operate completely sealed from the environment. Methane produced in the process is collected and added to the local gas grid. Additionally, it was ensured that no venting of the facilities occurs at any time, eliminating the possibility of the methane peaks occurring due to short methane venting events during their operation.\\
The CF-IRMS measurements at the Geaomatikum also showed the large Hamburg port, with its associated industry park, has only a minor contribution to the methane mixture in Hamburgs air.\\
Using the peak identification algorithm and identification criteria provided by \cite{Menoud.2021}, a difference in the isotope composition of the methane between the background and the methane peaks was observed. The peaks showed a more substantial influence by anthropogenic sources, like fossil fuel combustion. indicating that many of those peaks originated from methane plumes produced by Anthropogenic emitters. While the background shows a generally strongly influenced methane composition by natural production mechanisms such as wetlands and agriculture.\\
A difference between the smaller and prominent methane peaks could be observed by applying stricter peak identification criteria to the measurement timeline. The smaller peaks generally have more anthropogenic attributions, while the prominent peaks originate from microbial and wetland methane emitters. The prominent pieces showed an isotopic signature of $\delta$13C = -60.3$\permil$ and $\delta$D = -298$\permil$, putting them firmly in the microbial production mechanisms, most likely by wetlands and water body. \\
This indicated that the extremely high dry air mole fraction of methane peaks with up to 4500 ppb originate from the Elbe and its connected water bodies with their riverbanks and wetlands.\\
A correlation of water quality parameters such as water temperature, oxygen concentration and saturation, turbidity, UV absorption and pH level with the methane concentration at specific wind directions gave further indications pointing to the Elbe responsible for the prominent methane peaks. Research at comparable rivers such as the Themse in London by \cite{Zazzeri.2017} also showed that the methane emissions of a river could significantly contribute to the isotope signature mixture in large cities' air.\\
A purpose build Gaussian plume time reversed transport model showed that the origin of the methane peaks most likely lay in a radius of around 5 to 12 km from the Geomatikum. The transport model showed that the most likely emission location was in the waterbodies in and around the city of Hamburg. Those include the wetlands to the South-West, the port in the South and the channels, fleets, and harbours in the historic city centre. The transport model has the limitation that the wind data measured at one of the four locations in Hamburg is used. If a particle during the modelling travels too far from the wind measurement location, it could experience a different wind system not detected at the measurement location. This is particularly interesting when the particles leave the city borders. It is well known that a city’s climate can differ largely from its surrounding due to Topography and temperature differences. Unfortunately, no weather station with the same degree of standardisations as seen from the DWD and Uni Hamburg is available near the city. So the assumption of uniform wind in the region was taken. This limits the reliability of the transport model for large distances that exceed the city borders and its direct surroundings.\\
The occurrence of methane peaks could also be linked to the dropping of the water level in the Elbe due to the tidal cycle. As shown by \cite{Harrison.2017}, dropping the water level increases the methane concentration dissolved in the Elbe water. \cite{Matousu.2017} showed that the dropping in water level in water reservoirs with high sediments and pollution causes a significant release of methane into the atmosphere due to the hydrostatic water pressure reduction with the drop of the water column height. \cite{Matousu.2019}  showed that the Elbe forms methane production hotspots in heavily human-altered and impoundment river sections, including the region of Hamburg.\\
With the acquired correlations and modelling together with the resource found in the literature, it can be concluded that the methane peaks observed in the city centre of Hamburg originate from the Elbe.\\
A complex interplay of many factors enables this behaviour. The river is fertile for methane production, mainly when high pollution and low water quality are observed. The methane can accumulate in the sediments over time, as natural methane reduction methods are disturbed due to the heavy impoundment within the city region. The tidal cycle of the Elbe is quite large for a river due to its connection to the Waddensea and the North Sea. While methane-rich water from the Wadden Sea is also flushed upstream by the tides. The fast drop in the water level that even allows for the river to run dry in certain regions is the main catalyst for releasing methane into the atmosphere. The low water level reduces the pressure allowing bubbles to form and travel to the surface, while the short water column height doesn’t allow for sufficient methane oxidation. The regions where the sediments are exposed to the air may also release significant methane. While the resetting of the sediments due to the suddenly accelerated water flow over the sediments due to the low water may also reintroduce organic matter to the sediments, which is known to accelerate methane production \cite{Bednarik.2019}. \\
The suddenly released methane is then transported away from the Elbe by the wind. Depending on the directions and speed of the wind, this can produce a very sudden and high methane concentration peak in the city's air. It is assumed that the Geomatikum is located conveniently to observe such peaks. At the same time, a measurement location closer to the river would probably result in more regularly sized and frequent peaks.\\
That the  Elbe is an unaccounted methane emitter has also been shown by the Inverse Bayesian modelling for the FTIR measurement. As \cite{Forstmaier.2023} demonstrates, the modelling could significantly be improved when the Elbe is accounted for in the Prior. The tidal cycle and its resulting variation in methane emission from the Elbe are unfortunately not accounted for in this correction. This could further improve the model.
